\clearpage
\section*{Введение}
\addcontentsline{toc}{section}{\hspace{7mm}Введение}

Язык программирования Java на сегодняшний день является одним из самых популярных инструментов
разработки программного обеспечения. \footnote{\url{http://langpop.com/}}

Во многом своей популярности он обязан платформе JRE (Java Runtime Environment), включающей в себя
стандартную библиотеку языка и виртуальную машину Java (JVM), на которой исполняется скомпилированный код.
Именно наличие последнего пункта значительно облегчает задачу разработчика, решая такие непростые
проблемы,  как кросплатформенность, безопасность, управление памятью и отчасти производительность ПО.

Однако сам язык Java обладает рядом существенных недостатков: многословный и ``бедный'' синтаксис,
слабый вывод типов, развитие языка заторможено необходимостью сохранять обратную совместимость.

В связи с этим на рынке появляются новые языки, преодолевающие эти недостатки, но также
компилирующие исходный код программ в байткод для JVM и сохраняющие таким образом преимущества платформы.
Примером таких языков могут послужить Groovy, Scala, Kotlin, Clojure, Ceylon.

Поскольку для решений, базирующихся на платформе Java, нередко критичным фактором является
производительность и размер байткода приложения, то при сравнении этих языков в том числе следует
учитывать эти параметры кода, генерируемого их компиляторами.

Kotlin разрабатывается в компании JetBrains с 2010 года, а выпуск первой версии запланирован на середину 2015 года.

В рамках данной работы будет произведен анализ производительности байткода генерируемого его транслятором,
а также описаны решения, принятые для устранения найденных недочетов, такие как:
\begin{itemize}
    \item Устранение избыточного боксинга.
    \item Оптимизация генерации кода для сочетания операторов безопасного вызова и ``Elvis''.
    \item Оптимизация генерации оператора ``when'' для целочисленных констант, строк и классов-перечислений.
    \item Удаление мертвого кода.
    \item Решение проблемы с нарушением условий для <<On-stack-replacement>>.
\end{itemize}

%Квалификационная работа состоит из двух глав.

%В первой главе рассмотрены основные понятия предметной области, а также детализированы цели и задачи данной работы.

%Во второй рассмотрены уже существующие решения и приведены детали реализации полученной в рамках данной работы системы.

%В заключении описаны результаты работы и сформулированы перспективы дальнейшего развития системы.

\clearpage

\section*{Постановка задачи}

Главной целью данной работы --- улучшение производительности байт-кода, генерируемого компилятором
Kotlin.

Для ее достижения необходимо было решить следующие задачи:
\begin{itemize}
    \item Изучение подходов к производительности кода, реализованных в компиляторах других
    языков программирования для платформы Java.

    \item Измерение производительности байт-кода, генерируемого компилятором для различных
    синтаксических конструкций.

    \item Анализ, трактовка и классификация найденных в рамках стадии измерения проблем.

    \item Исследование оптимизаций, проводимых современными реализациями виртуальной машины
    Java, необходимое для определения критичности найденных проблем в генерируемом коде.

    \item Внедрение в компилятор изменений, способствующих решению найденных недостатков в коде
    и измерение производительности после этих изменений.
\end{itemize}

\clearpage
