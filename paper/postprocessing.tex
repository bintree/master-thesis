\subsection{Постобработка байт-кода}
\label{section:postp}
В рамках анализа результатов запусков бенчмарков был выявлен класс проблем, которые было бы сложно
решать в рамках фазы кодогенерации, а именно избыточный боксинг (см. \ref{section:inline:bm}) и
нарушения условия для выполнения OSR (\ref{section:osr:bm}).

Основная сложность заключается в том, что корень этих проблем лежит во встраивании функций ---
механизме, который работает с уже сгенерированным байт-кодом (встраиваемой функции), причем
семантика встраиваемого кода хотя бы в виде исходников для подсистемы встраивания недоступна,
что затрудняет оптимизации на этой фазе, и, так или иначе приводит к необходимости постобработки
байт-кода.

Надо отметить, что избавление от артефактов встраивания в отдельной фазе компиляции ни в коей мере
не является чем-то новым и обсуждается, как в известной литературе\cite{Muchnick}, так и в работе
про оптимизации в Scala\cite{ScalaDragos}.

\subsubsection{Анализ потока данных}
Анализ потока данных --- являются наиболее
часто используемым подходом при постобработке сгенерированного кода и при реализации
различных вариантов статического анализа.

С его помощью можно проверять или выводить те или иные утверждения о программном коде в самых разных
его аспектах.

Здесь описывается общий каркас для алгоритмов анализа потока данных, изложенный частично в \cite{Muchnick}
и \cite{Nielson}.

Основная суть этого подхода заключается в представлении процедуры в виде графа, где вершины ---
базовые блоки --- отдельные инструкции промежуточного языка или непрерывно исполняемые участки
программы, а ребра --- возможные переходы между ними.
Ориентация ребер зависит от вида анализа, который бывает \textit{прямым} или \textit{обратным}.

Еще одним важным элементом анализа является решетка
$$\mathbb{T} = (\mathbb{T}, \sqsubseteq, \bigsqcup, \bigsqcap, \top, \bot)$$
определяющая состояние программы в каждом базовом блоке.

Операция $\bigsqcup$ определяет состояние, возникающее в результате слияния других состояний.

Нередко в качестве состояния программы подразумевают состояние ее ячеек памяти, тогда в качестве
решетки удобно взять вектор $\mathbb{T} :: = StackState = (\mathbb{X}^v)$, где $v$ --- число ячеек
памяти, а $\mathbb{X}$ --- некоторая другая решетка.

Также для каждого базового блока задают монотонную трансфер-функцию
$transfer_b :: \mathbb{T} \to \mathbb{T}$, отражающую семантическое влияние блока на состояние программы
при его исполнении.
Монотонность трансфер-функции необходима для сходимости алгоритма и чаще всего очевидна.

Далее строится система уравнений --- по два на каждый блок $b$:
$$in_b: \mathbb{T} ::= \bigsqcup \{out_x\ |\ x \in pred_b\}$$
$$out_b: \mathbb{T} ::= transfer_b(in_b)$$

Также эту систему удобно представить в виде векторной функции: $F :: \mathbb{T}^{2n} \to \mathbb{T}^{2n}$

Решение этой системы так или иначе отвечает на вопрос, какие состояния в принципе могут быть
актуальны в различных точках программы.
Однако чаще всего интересует ответ на вопрос о минимальном решении, т.к. обычно решетка строится
так, что меньшие элементы являются условно более точными.
Его определяют как:
$$lfp(F) = \bigsqcap Fix(F) =  \bigsqcap\{ x | F(x)= x\}$$

Из определения очевидна минимальность полученного вектора из тех, которые в принципе могут быть
решениями, а из теоремы Тарского следует что результат будет неподвижной точкой, а кроме
того в виде следствия дается алгоритм для его вычисления: $lfp(F) = F^n(\bot)$, где $n \to \infty$.

Причем для конечных решеток с учетом монотонности $F$ очевидна сходимость алгоритма за конечное
число шагов.

\subsubsection{Архитектура подсистемы}
В рамках реализации подсистемы постобработки для работы с байт-кодом, и для реализации алгоритмов
анализа потока данных использовались существенные части библиотеки ASM\fu{http://asm.ow2.org/}.

Она предоставляет набор инструментов для работы с class-файлами, байт-кодом методов и каркас,
реализующий общую часть алгоритма анализа потока данных.

ASM предлагает весьма удобную абстракцию для представления байт-кода метода --- класс
``MethodNode'', в котором с множеством инструкций можно обращаться, как с обычным изменяемым
списком, а отдельные инструкции представлены в виде классов с удобным интерфейсом доступа
к их аргументам. Например, инструкции переходов хранятся как объекты класса ``JumpInsnNode'',
в котором в качестве поля содержится ссылка на инструкцию, куда может быть произведен переход.

\paragraph{Каркас алгоритма анализа потока данных}
Также, как уже было сказано выше, библиотека предоставляет базовую реализацию алгоритмов, которые
в частности можно использовать и для анализа потока данных и потока управления в виде каркаса.

Для его использования разработчику следует:
\begin{itemize}
    \item Создать класс, реализующий интерфейс ``Value''. Объекты этого класса как раз и будут
    элементами решетки $\mathbb{X}$, описанной в предыдущем разделе.

    \item Реализовать интерфейс ``Interpreterer'', методы которого как раз описывают трансфер-функцию
    для каждого типа инструкций.
    Кроме того, здесь же необходимо определить операцию $\bigsqcup$ для двух элементов
    решетки, реализовав метод ``merge''.

    \item После чего создается объект класса ``Analyzer'', реализующий логику поиска наименьшего
    решенияи параметризуется конкретными реализациями ``Value'' и ``Interpreter''.

    Метод ``analyze'' принимает объект ``MethodNode'' и возвращает массив из объектов ``Frame'',
    каждый из которых содержит информацию о состоянии стек-фрейма для каждой из инструкций метода.

    В классе ``Frame'' реализованы методы ``getLocal''/``getStack'', возвращающие элемент решетки,
    находящийся в полученном решении в конкретной локальной переменной или на определенном сдвиге
    от вершины стека.
\end{itemize}

Следует, однако, уточнить, что предложенный в библиотеке каркас обладает рядом недостатков:
\begin{itemize}
    \item Для его корректной работы требуется точное вычисление максимального размера стека
    в методе, а также числа используемых переменных.

    Для автора данной работы остаются загадкой причины, по которым вычисление этих величин
    не производится в рамках анализа самим каркасом.
    Причем единственное место в библиотеке, где эта логика реализована --- подсистема генерации
    бинарного представления class-файла.

    Таким образом, у разработчиков остается два варианта: искусственно генерировать бинарное
    представление метода, откуда можно получить эти величины, или в своем коде дублировать логику
    их вычисления, что в общем случае не тривиально.
    Автором был выбран второй подход, так как он значительно менее ресурсоемок.

    \item Еще одной существенной проблемой каркаса является его недостаточная гибкость:
    \begin{itemize}
        \item Аналогам трансфер-функций доступен не весь стек-фрейм, а только та его часть,
        которая относится к конкретной инструкции, чего не всегда бывает достаточно.

        \item Для некоторых видов инструкций, таких как ``POP'' --- просто снимающих значение
        с вершины стека --- не предусмотрены трансфер-функции, и их обработку можно произвести
        только в формате дополнительного анализа, запустив его после основного.

        \item Элементы стека в классе ``Frame'' неизменяемыми, единственное, что возможно ---
        удалять элемент с вершины стека, и добавлять новый.
    \end{itemize}

    \item Каркас предоставляет возможность только для <<прямого>> анализа, таким образом его
    невозможно использовать, например, для определения интервалов времени жизни переменных.
\end{itemize}

Однако для решения задач анализа, поднятых в рамках данной работы, функциональности библиотеки
в целом было достаточно, поэтому описанные далее алгоритмы так или иначе были реализованы
с помощью этого каркаса.

\paragraph{Внедрение в компилятор}
Одним из тривиальных, но тем не менее важных пунктов работы, стала задача внедрения в компилятор
подсистемы постобработки байт-кода.

Сама разработанная подсистема состоит из двух частей:
\begin{itemize}
    \item Множество алгоритмов, так или иначе реализующих непосредственно логику изменения
    байт-кода.

    Каждый из них должен быть представлен в виде класса, реализующего абстрактный класс
    ``MethodTransformer'', содержащий единственный абстрактный метод ``transform'', принимающий
    ``MethodNode'' в качестве аргумента, в котором и должна быть описана содержательная часть
    алгоритма.

    Таким образом, эта часть состоит из списка алгоритмов ``MethodTransformer'', которые в принципе
    должны быть независимы друг от друга, однако на деле важным оказывается порядок, в котором
    эти алгоритмы применяются к каждому из методов.

    \item Часть, взаимодействующая с подсистемой кодогенерации.
    Здесь можно вкратце сказать от том, что последняя оперирует с байт-кодом с помощью абстракции
    все той же библиотеки ASM --- MethodVisitor, и для добавления следующий инструкции в байт-код
    вызывается тот или иной метод этой абстракции.

    Поэтому для того, чтобы перед непосредственно записать в class-файл байт-код метода,
    выполнялась его пост-обработка необходимо было с помощью шаблона проектирования
    <<декоратор>>\cite{Gamma} реализовать свой вариант ``MethodVisitor'', аккумулирующий байт-код
    в ``MethodNode'', и при вызове ``endVisit'', запускающий процесс трансформации, результат
    которого передается в декорируемый ``MethodVisitor''.

    Этой реализацией необходимо заместить существующую, которая создается при создании нового метода
    в классе ``ClassBuilder''.
\end{itemize}

\subsubsection{Удаление избыточного боксинга}
\label{section:boxing}
В рамках раздела \ref{section:inline:bm} было выявлено одно из проблемных мест при встраивании
функций, а именно наличие избыточного боксинга.

Для начала необходимо определить и зафиксировать понятия <<боксинга>> и его <<избыточности>>.

\textit{Боксингом} назовем получение объекта одного из классов-оболочек для хранения примитивных
типов (см. раздел \ref{section:lambda}) по имеющемуса примитивному значению.
Чаще всего это происходит в результате вызова статического метода ``valueOf'' у этих классов.

\textit{Избыточной} назовем операцию боксинга, результат которой --- объект класса-оболочки ---
используется только для условно <<чистых>> операций, т.е. таких, которые могут быть
безопасно заменены для работы с примитивами, либо удалены:
\begin{itemize}
    \item Сохранение или загрузка значения в локальную переменную.
    \item Cтандартные операции на стеке: POP/DUP.
    \item Инструкции уточняющие информацию о значении на вершине стека: IFNULL, IFNONNULL,
    CHECKCAST, INSTANCEOF.
    \item Вызовы методов, возвращающие хранимое значение примитивного типа, например
    ``Integer.intValue()''.
    \item Вызовы методов для конвертации запакованного значения к типам других примитивов,
    например, метод ``Integer.longValue()''.
\end{itemize}

Суть этих требований проста:
\begin{itemize}
    \item Если значение, полученное в результате боксинга используется в рамках других инструкций,
    например, вызов метода ``hashCode'', то очевидно, что операция упаковки неизбежно должна иметь
    место до этого вызова.

    \item С другой стороны все перечисленные операции действительно либо могут быть удалены, как
    в случае с инструкцией IFNULL --- так как известно, что запакованное значение не может быть
    равно нулевой ссылке, либо могут быть заменены на аналогичные операции с примитивами:
    ASTORE $\to$ ISTORE, ``Integer.longValue()'' $\to$ I2L.
\end{itemize}

Также необходимо, чтобы при любом пути исполнения метода не возникало ситуации, когда в том же
слоте стек-фрейма, где хранится запакованное значение, которое потенциально <<избыточно>>
не могло теоретически оказаться:
\begin{itemize}
    \item Другого объекта ссылочного типа. Как минимум причиной для такого требования может служить
    спецификация виртуальной машины, запрещающая разделение слота значениями различной природы.
    Кроме того, в результате смешивания значений должен наблюдаться специфический результат,
    например, в коде ниже в зависимости от условия должна сохраняться возможность исключения
    ``NullPointerException'':
    \begin{pyglist}[language=kotlin]
        val x = if (cond) Integer.valueOf(1) else null
        x.intValue() // Possible NullPointerException
    \end{pyglist}

    \item Объекта полученного в результате боксинга другого типа, например, Integer и Long.

    Это связано с тем, что при потенциальной возможности такого слияния, заменить значения на
    примитивы не удастся опять же в связи с тем, что это запрещено спецификацией.

    \item Значения, полученного в результате боксинга, которое не может быть удалено.
\end{itemize}

Таким образом, боксинг можно назвать \textit{избыточным} тогда и только тогда, когда полученное
значение используется только как операнд для <<чистых>> инструкций и любое значение, с которым
первое значение может разделять место в стек-фрейме является запакованным значением такого же
типа, боксинг которого можно назвать \textit{избыточным}.

Отметим, что определение получилось рекурсивным.

\paragraph{Специализированные итераторы}
В Kotlin определены специализированные версии итераторов для каждого из примитивных типов:
``IntIterator'', ``FloatIterator'' и т.д.

Каждый из них является абстрактным классом, требующим реализации метода вида ``nextT'', например
``nextInt'':
\begin{pyglist}[language=kotlin]
    public abstract class IntIterator : Iterator<Int> {
        override final fun next() = Integer.valueOf(nextInt())
        public abstract fun nextInt(): Int
    }
\end{pyglist}

Важно отметить, что такие итераторы реализуют более общий параметрически-полиморфный интерфейс
``Iterator<T>'', где возникает необходимость в боксинге значения.

Наиболее часто встречающимися примерами реализации таких итераторов являются итераторы
по числовому отрезку. Синтаксически выраженные как $(low..high)$, при трансляции в JVM они
становятся объектами классов ``<T>Range'', где T --- один из числовых примитивов, которые можно
использовать как обычные неизменяемые коллекции, например:
\begin{pyglist}[language=kotlin]
    (1..100).count { it % 2 == 0 }
\end{pyglist}

Причем в качестве итераторов возвращаются как раз объекты специализированных классов, однако
большая часть библиотечных функций, включая ``count'' из примера, работают с ними, как с обычными
итераторами, вызывая метод ``next'', который также можно считать операцией боксинга наравне
с вызовами ``valueOf''.

Таким образом, необходимо также анализировать является ли объекты, ссылки на которые хранятся
в стек-фрейме, специализированными итераторами или объектами классов "<T>Range".

\paragraph{Определение решетки}
Таким образом, множество элементов решетки можно задать так:
$$\mathbb{X} = \{\bot = Nothing \sqsubseteq  Primitive, Boxed, Range, PrimitiveIterator \sqsubseteq \top = Object \}$$
Здесь
\begin{itemize}
    \item $\top = Object$ --- произвольное значение ссылочного типа.
    \item $\bot = Nothing$ --- неинициализированное значение, в частности необходимое для полноты
    решетки.
    \item $Primitive$ --- значение примитивного типа.
    \item $Boxed$ --- ссылочное значение, полученное в результате боксинга.
    \item $Range$ --- ссылочное значение, являющееся объектом вида $(low..high)$.
    \item $PrimitiveIterator$ --- ссылка, указывающая на специлизированный итератор.
\end{itemize}

Операция $\bigsqcup$ для двух элементов задается так:
$$x \bigsqcup y =
\begin{cases}
y & \text{если } x == \bot \\
x & \text{если } y == \bot \\
x & \text{если } x == y \\
Object & \textit{Иначе}
\end{cases}
$$

Под равенством элементов подразумевается, как эквивалентность элементов решетки, так и равенство
содержимых типов (для Boxed, Range, PrimitiveIterator).

Трансфер-функции для большей части случаев тривиальны, так как порождены семантикой инструкций.
Исключение составляют вызовы методов, продуцирующие новые значения специфичных типов:
\begin{itemize}
    \item \textit{transfer(INVOKESTATIC, WRAPPER\_CLASS, "valueOf(T)") = Boxed}
    \item \textit{transfer(INVOKESPECIAL, RANGE\_CLASS, "<init>") = Range}
    \item \textit{transfer(INVOKEINTERFACE, "iterator()") = PrimitiveIterator}, если аргумент вызова есть $Range$.
    \item \textit{transfer(INVOKEINTERFACE, "next()") = Boxed}, если аргумент вызова есть $PrimitiveIterator$.
\end{itemize}

Следует уточнить, что полученные трансфер-функции не являются функционально-чистыми, так как
в момент анализа аккумулируют список Boxed-значений, изменяют информацию о <<чистоте>> их
использования в случае <<неудачного>> слияния и так далее.

Таким образом, в результате реализованный автором алгоритм анализа возвращает список Boxed-значений,
которые согласно определению выше являются избыточными, а также список инструкций, для
которых каждое из значений является операндом.

\paragraph{Адаптация инструкций}
После выяснения множества чистых Boxed-значений, необходимо адаптировать оперирующие
с ними инструкции для работы с аналогичными примитивами.

Например, удалить операции ``CHECKCAST'', производящие принадлежность объекта классу, если результат
в любом случае положителен (например, ``CHECKCAST Number'' применяемый к Integer), заменить
инструкции доступа к переменным ссылочного типа на соответствующие инструкции, работающие
с примитивами и т.д.

Также необходимо изменять информацию об изменении типа в таблице локальных переменных.
Это может быть несколько нетривиально в случае, когда тип меняется со ссылочного на примитив,
занимающий два слота памяти: ``long'' или ``double''.
В этом случае необходимо сдвигать индексы переменных, имеющих большие номера, причем изменять надо
как в таблицу, так и инструкции байт-кода.

\paragraph{Недостатки решения}
Предложенное решение обладает рядом незначительных недостатков.

Например, если запакованное значение в большинстве случаев используется только для получения
хранимого значения, и только в одном месте является операндом <<грязной>> инструкции, данный
алгоритм ничего не изменит.

Автор статьи про оптимизации компилятора Scala\cite{ScalaDragos} предлагает альтернативное
решение --- перед боксингом сохранять примитив во временную переменную, и использовать ее
значение каждый раз вместо процедуры <<распаковки>>.

Причем сама операция боксинга или эта временная переменная будет удалена последующим анализом
времени жизни переменных в случае их неиспользуемости.

Однако в статье не был подробно разобран момент слияния двух различных Boxed-значений:
кажется единственным вариантом в момент слияния создавать еще одну переменную, по аналогии с тем,
как это происходит с $\varphi$-функциями в $SSA$\cite{Muchnick}, что существенно усложняет
трудоемкость анализа, а кроме того добавляет потенциально лишние инструкции в байт-код.

Для оценки того, насколько часто в обычном коде встречаются такие сложные случае, был реализован
оптимистичный анализ, детектирующий любые случаи боксинга, результат которых хотя бы раз
используется в методе для распаковки значения, т.е. потенциально избыточные.

Результат был запущен на нескольких проектах, написанных на Kotlin: реализации задач Project Euler,
Kannotator, Kara.
Всего было найдено около 600 искомых операций боксинга, из которых только 7 не могли быть удалены
описанным выше алгоритмом.
Таким образом, предложенное решение позволяет избежать боксинга в большинстве найденных случаев.

\paragraph{Результаты}
После внедрения описанного решения в компилятор были проведены повторные измерения
производительности на бенчмарках со встраиваемыми функциями. Результаты приведены в таблицах
\ref{bm:count:opt}, \ref{bm:filter:opt} и \ref{bm:fold:opt}.

\begin{table}[h]
\begin{center}
\begin{tabular}{|c|c|c|c|c|} \hline
Размер & Эталон & До & После & Ускорение (раз) \\ \hline
100 & 87.968 нс & 348.255 нс & 88.63 нс & 3.929\\ \hline
10000 & 18.401 мкс & 78.05 мкс & 18.655 мкс & 4.184\\ \hline
1000000 & 3.715 мс & 8.565 мс & 3.711 мс & 2.308\\ \hline
\end{tabular}
\caption{Сравнение производительности вызова ``count \{ it \% 2 == 0 \}'' до и после оптимизаций}
\label{bm:count:opt}
\end{center}
\end{table}

\begin{table}[h]
\begin{center}
\begin{tabular}{|c|c|c|c|c|} \hline
Размер & Эталон & До & После & Ускорение (раз) \\ \hline
100 & 598.23 нс & 711.114 нс & 482.667 нс & 1.473\\ \hline
10000 & 98.655 мкс & 121.1 мкс & 97.938 мкс & 1.236\\ \hline
1000000 & 11.868 мс & 13.68 мс & 11.292 мс & 1.212\\ \hline
\end{tabular}
\caption{Сравнение производительности вызова ``filter \{ it \% 2 == 0 \}'' до и после оптимизаций}
\label{bm:filter:opt}
\end{center}
\end{table}

\begin{table}[h]
\begin{center}
\begin{tabular}{|c|c|c|c|c|} \hline
Размер & Эталон & До & После & Ускорение (раз) \\ \hline
100 & 27.832 нс & 594.34 нс & 27.928 нс & 21.281\\ \hline
10000 & 2.773 мкс & 100.181 мкс & 2.776 мкс & 36.092\\ \hline
1000000 & 301.701 мкс & 11.28 мс & 304.423 мкс & 37.054\\ \hline
\end{tabular}
\caption{Сравнение производительности вызова ``fold(0) \{ sum, x -> sum + x \}'' до и после оптимизаций}
\label{bm:fold:opt}
\end{center}
\end{table}

\subsubsection{Избыточные проверки на равенство null}
Еще один этап постобработки --- удаление избыточных проверок на равенство null.

Несмотря на то, что наличие таковых избыточных проверок в пользовательском коде нетипично, они
могут появляться при встраивании --- когда встраиваемая функция написана с расчетом на то, что
определенное значение может быть нулевой ссылкой, однако после встраивания можно доказать, что
это невозможно.

Также это может быть необходимо после генерации оператора приведения `as`, семантика которого
обязывает генерировать проверку на равенство null.

Рассматриваемый вид оптимизаций направлен не столько на увеличение производительности байт-кода,
так как сами по себе эти проверки не слишком трудоемки, хотя и могут отрицательно влиять на
предсказатель переходов ЦПУ, сколько для уменьшения размера получаемого байт-кода: удаление
таких проверок может делать недостижимыми большие участки кода.

\paragraph{Анализ}
Анализ предмет вывода возможности равенства нулевой ссылке различных значений
не является чем-то существенно новым в мире статического анализа JVM-байт-кода.
В данной работе реализована простая версия таких алгоритмов, также как и предыдущее решение,
основанная на фреймворке для анализа потока данных из библиотеки ASM.

Здесь используется похожая решетка значений, как и в случае анализа из предыдущего раздела:
$$\mathbb{X} = \{\bot = Nothing \sqsubseteq  Primitive, NotNull \sqsubseteq \top = Object \}$$

Здесь \textit{NotNull} --- значение про которое доказано, что оно содержит ссылку, не являющуюся
нулевой ссылкой.

Операция $\bigsqcup$ задается аналогично, с тривиальным уточнением:
$$x \bigsqcup y = NotNull \Leftrightarrow x = NotNull \land y = NotNull$$

Трансфер-функции снова тривиальны за исключением тех, которые порождают NotNull-значения:
\begin{itemize}
    \item ANEW, ANEWARRAY --- создающие новый объект или массив.
    \item LDC --- загружающая константу из пула констант.
    \item Инструкции, так или иначе, порождающие боксинг, гарантируют, что полученный объект
    не равен null.
\end{itemize}

Последний пункт порождает необходимость абстракции кода, вычисляющего множество Boxed-значений,
для его переиспользования в обоих случаях.

В свою очередь удаление всех таких избыточных проверок упрощает трансформацию кода для
удаление избыточного боксинга.

\paragraph{Трансформация байт-кода}
После получения результатов анализа для каждой инструкции IFNULL/IFNOTNULL в случае, если можно
доказать, что значение на вершине стека не равно null, то:
\begin{itemize}
    \item Значение снимается с вершины стека, добавление инструкции POP перед текущей.

    Это необходимо, чтобы размер стека после исполнения остался таким же, как и в случае
    с исходной инструкцией.

    \item Инструкция IFNULL просто удаляется, так как такой переход в любом случае не будет
    выполнен, а инструкция IFNOTNULL заменяется на безусловный переход --- GOTO к той же метке,
    что и в исходной инструкции.

    \item Дополнительная микрооптимизация состоит в том, что если инструкция перед POP не имеет
    посторонних эффектов при своем исполнении, например, загрузка локальной переменной --- ASTORE,
    или копирование значения на вершине стека --- DUP, то можно, не добавляя искусственной инструкции
    POP, просто удалить предыдущую.
\end{itemize}

\subsubsection{Удаление недостижимого кода}
Один из результатов оптимизации, описанной в предыдущем разделе, состоит в порождении веток
недостижимого или <<мертвого>> кода.

В принципе мертвый код может порождаться и в других случаях, например, с точки зрения системы типов
если тело функции состоит из вызова другой функции, имеющей тип ``Nothing'', то можно опустить
return-оператор, так как считается, что любое выражение, имеющее такой тип, не может завершиться.

Однако с точки зрения виртуальной машины такой метод выглядит так, как будто в нем есть ветка
исполнения, в которой не вызывается инструкция RETURN, в связи с чем система кодогенерации
добавляет искусственную инструкцию RETURN в конец каждого метода, причем в большинстве
случае такие инструкции недостижимы.

Формальное определение недостижимости инструкции просто выводится из представления кода в виде
графа --- базовый блок, или инструкция называется недостижимой, если не существует пути
ведущего в этот блок из точки входа в процедуру.

Самый простым с точки зрения реализации способ разметить достижимость кода в рамках библиотеки ASM
является запуск стандартного анализатора, поставляемого вместе с библиотекой.

Гарантируется, что когда в полученном массиве фреймов элемент равен нулевой ссылке, то
соответствующая ему инструкция недостижима.

Найденные таким образом недостижимые инструкции просто удаляются из списка ``MethodNode''.

\paragraph{Результаты}
Результаты подобных оптимизаций достаточно сложно оценивать.
Понятно, что можно написать код, большая часть которого окажется мертвой и заявить об уменьшении
размера class-файла в десятки раз.

Наиболее адекватным кажется измерять сокращение размера получаемых class-файлов в произвольном
<<промышленном>> проекте, написанном на Kotlin.

В качестве такого образцового проекта был выбран Kannotator\fu{https://github.com/JetBrains/kannotator}.
Суммарный размер class-файлов без оптимизаций составляет 3.4 МБ, после оптимизаций --- 3.3 МБ.

Таким образом, сокращение размера байт-кода составило до 3\%.
Следует отметить, что подобного порядка результаты наблюдаются и после фазы оптимизации
в ProGuard\fu{http://proguard.sourceforge.net/results.html}.

\subsubsection{Замена на стеке}
\label{section:osr:opt}
В разделе \ref{section:osr:bm} была выявлена ситуация, когда встраивание вызовов функций, содержащих
циклы, может приводить к серьезной деградации производительности --- до нескольких сотен раз ---
из-за того, что JIT-компилятор Hotspot не производит замену на стеке для таких случаев.

Для ликвидации этой проблемы было реализовано простое, и вместе с этим действенное решение:
перед каждым встроенным вызовом стек следует полностью освобождать, сохраняя значения во временные
переменные, и восстанавливая после вызова.

Более подробно решение состоит из следующих пунктов:
\begin{itemize}
    \item Подсистема, производящая встраивание вызовов, отмечает в байт-коде интервал
    на котором оно происходит.
    Наиболее удобным способом сделать это --- добавить в качестве маркеров начала и конца
    интервалов вызовы специальных методов с пустым телом.

    \item В подсистеме постобработки производится анализ таких меток в каждом методе.

    В первую очередь проверяется их корректность --- последовательно расположенные маркеры должны
    образовывать сбалансированную <<скобочную>> последовательность.
    Это условие может нарушаться в случае, когда часть из них удаляется при встраивании.
    В случае нарушения условия оптимизация не производится.

    Иначе строится множество интервалов встраивания.

    \item После этого производится анализ байт-кода: необходимо определить кортеж из типов
    значений, расположенных на стеке в начале каждого из интервалов.

    Один из стандартных интерпретаторов, поставляемых с библиотекой ASM, позволяет решить эту
    задачу.

    \item Далее точки из интервалов обрабатываются в порядке их появления в байт-коде:
    \begin{itemize}
        \item В начале интервала все значения, хранящиеся в стеке,
        сохраняются во временные переменные, соответствующие инструкции добавляются перед маркером.
        \item В конце интервала значения загружаются из переменных в обратном порядке,
        восстанавливая, таким образом, состояние стека.
        После этого переменные отмечаются, как освобожденные, и могут быть использованы
        в следующем начале интервала.

        Следует уточнить, что если встраиваемая функция возвращала какое-то значение, то оно
        оказывается на вершине стека как раз перед маркером конца встраивания, при этом
        в исходном байт-коде оно должно располагаться после сохраненных значений.
        Для этого это значение также необходимо сохранить во временную переменную, и загрузить
        уже после сохраненных ранее.
    \end{itemize}

    \item После прохода алгоритма маркеры удаляются из списка инструкций.
\end{itemize}

Таким образом, полученное решение гарантирует отсутствие ситуации с непустым стеком перед встроенным
вызовом, и, как следствие, перед любым циклом.

\paragraph{Результаты}
После внедрения решения в компилятор время работы бенчмарка значительно приблизилось к эталону
(см. таблицу \ref{bm:foldOSR:opt}).

\begin{table}[h]
\begin{center}
\begin{tabular}{|c|c|c|c|c|} \hline
Размер & Эталон & До & После & Ускорение (раз) \\ \hline
100 & 27.832 нс & 595.958 нс & 28.065 нс & 21.235\\ \hline
10000 & 2.773 мкс & 95.095 мкс & 2.796 мкс & 34.008\\ \hline
1000000 & 301.701 мкс & 114.624 мс & 302.123 мкс & 379.395\\ \hline
\end{tabular}
\caption{Сравнение производительности вызова ``foldBenchmarkOSR'' до и после оптимизаций (версия Hotspot 1.7)}
\label{bm:foldOSR:opt}
\end{center}
\end{table}
