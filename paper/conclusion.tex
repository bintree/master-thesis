\clearpage
\section*{Заключение}
\addcontentsline{toc}{section}{\hspace{7mm}Заключение}

В рамках данной работы достигнуты следующие результаты:

\begin{enumerate}
    \item Автором было произведено исследование подходов, применяемых для оптимизации
    производительности байт-кода в других компиляторах под платформу Java.

    \item Реализован набор бенчмарков, измеряющих эффективность байт-кода генерируемого
    компилятором Kotlin.

    \item Проведен анализ измерений, полученных в результате запусков бенчмарков, выявлены
    проблемные места.

    \item Также описаны изменения, исправляющие найденные недостатки и успешно внедренные в компилятор:
    \begin{itemize}
        \item Специализирована генерация байт-кода оператора when для целочисленных констант, строк и enum-классов \\
        \textit{(ускорение от 1.3 до 4.3 раз)}

        \item Оптимизирован процесс генерации сочетания Elvis + Safe-call \\
        \textit{(ускорение от 1.4 до 1.7 раз)}

        \item В компилятор добавлена подсистема постобработки байт-кода.
        \item Реализовано решение для удаления избыточного боксинга \\
        \textit{(ускорение от 1.2 до 37 раз)}
        \item Решена проблема с нарушением условия для <<замены на стеке>> \\
        \textit{(ускорение до 379.3 раз)}
        \item Реализован алгоритм для удаления избыточных проверок на равенство ссылок null
        \item Добавлено решение для удаления недостижимого кода
    \end{itemize}
\end{enumerate}
